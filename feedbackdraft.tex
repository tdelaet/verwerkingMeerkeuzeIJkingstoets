\section*{IJkingstoets burgerlijk ingenieur <editie>: algemene feedback}

In totaal namen <aantal> studenten deel aan de ijkingstoets burgerlijk ingenieur die aangeboden werd aan aspirant-studenten burgerlijk ingenieur aan de VUB, KU Leuven en UGent.  Hiervan waren er <G> geslaagd.  Zoals je kan zien in de onderstaande resultatenverdeling hebben heel wat deelnemers goed gepresteerd.  Daarnaast zijn er een aantal deelnemers met een lagere score, die zich best eens grondig bezinnen over hun studiekeuze en/of studieaanpak.

\begin{center}
\includegraphics[width=0.9\textwidth] {histogramGeheel}

Verdeling van de scores over de verschillende deelnemers van de ijkingstoets van <editie>
\end{center}

<N1>\% van de deelnemers haalde 18/20 of meer.

<N2>\% van de deelnemers haalde 16/20 of meer.

<N3>\% van de deelnemers haalde 14/20 of meer.

<N4>\% van de deelnemers haalde 12/20 of meer.

<N5>\% van de deelnemers haalde 10/20 of meer.

<N6>\% van de deelnemers haalde 7/20 of minder.

\vspace{1cm}


Hieronder staan de vragen, met telkens het juiste antwoord, het percentage dat deze vraag juist heeft beantwoord en het percentage dat deze vraag heeft blanco gelaten.

\vspace{1cm}
