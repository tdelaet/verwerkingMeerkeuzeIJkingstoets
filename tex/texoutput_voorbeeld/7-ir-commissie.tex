\documentclass[fleqn,a4paper,10pt,twoside]{article}
\usepackage[dutch]{babel}
%\usepackage{course}
\usepackage{graphicx}
\usepackage{fancyhdr}
\usepackage{wrapfig}
\usepackage[small,bf]{caption}
\usepackage[dutch]{babel}
\usepackage{amsmath}
\usepackage{amssymb}
\usepackage{amsfonts}
\usepackage{booktabs}
\usepackage{array}
%\usepackage{hyperref}
\usepackage{enumerate}
\usepackage{tikz}
\usetikzlibrary{through,calc,intersections}
%\usepackage{xcolor}
%\definecolor{KULblauw}{RGB}{26,67,121}
%\usepackage[urlbordercolor=KULblauw]{hyperref}
%\usepackage[urlbordercolor=white]{hyperref}

\setlength{\captionmargin}{1 cm}
\renewcommand{\captionfont}{\linespread{1} \small}

%\usepackage[thmmarks,framed,amsmath]{ntheorem}
\usepackage{ntheorem}
\usepackage{answers}
\theoremstyle{break}
%\setlength {\theorempostskipamount}{3cm}
\theoremheaderfont{\normalfont\bfseries}
\theoremsymbol{}
\theorembodyfont{\upshape}
 
\newtheorem{oefening}{Oefening} 
\newtheorem{soefening}{Samengestelde oefening}
\newtheorem{vraag}[oefening] {Vraag}

\Newassociation{opl}{Oplossing}{ans}
%\Newassociation{numopl}{Oplossing}{numans}
\newenvironment{numopl}[1]{\vspace{0.2cm} Oplossing: #1}{ }

\newcommand{\executeiffilenewer}[3]{%
 \ifnum\pdfstrcmp{\pdffilemoddate{#1}}%
 {\pdffilemoddate{#2}}>0%
 {\immediate\write18{#3}}\fi%
}


%\newenvironment{numopl}[1]{\vspace{0.2cm} Oplossing: #1}{ }
%%%%%%%%%%%%%%%%%%%%%%%%%%%%%%%
\newcommand{\ds}{\displaystyle}
\newcommand{\ex}{\vec e_x}
\newcommand{\ey}{\vec e_y}
\newcommand{\ez}{\vec e_z}
\newcommand{\cosec}{\mathrm{cosec}\,}

\newcommand{\R}{\ensuremath{\mathbb{R}}}
\newcommand{\C}{\ensuremath{\mathbb{C}}}
\newcommand{\Q}{\ensuremath{\mathbb{Q}}}
\newcommand{\Z}{\ensuremath{\mathbb{Z}}}
\newcommand{\graf}{\mathrm{graf}\,}

\newcommand{\be}{\begin{equation}}
\newcommand{\ee}{\end{equation}}
\newcommand{\ba}{\begin{array}}
\newcommand{\ea}{\end{array}}
\renewcommand{\labelenumi}{(\Alph{enumi})}
\renewcommand{\labelenumii}{(\Alph{enumii})}


 
%pagestyle{myheadings}
%\markboth{ \parbox{\includegraphics[height=1cm]{logo_ijkingstoets.jpg}} \footnotesize\sc{ IJkingstoets 1 juli 2013 - reeks1 - versie \today}} {\footnotesize\sc{IJkingstoets 1 juli 2013 - reeks1 - versie \today }}


%\includegraphics[height=1cm]{logo_ijkingstoets.jpg}

\setlength{\textwidth}{18 cm}
\setlength{\marginparwidth}{0 cm}
\setlength{\hoffset}{-1 cm}
\setlength{\voffset}{-3 cm}
\setlength{\oddsidemargin}{0cm}
\setlength{\evensidemargin}{0cm}
\setlength{\textheight}{26 cm}
\setlength{\parindent}{0cm}

\pagestyle{fancy}
\setlength{\headheight}{2cm}
\rhead{\parbox{14cm}{\footnotesize\sc{ IJkingstoets burgerlijk ingenieur 14 september 2015 - reeks 1 - p. \thepage
%\newline voorlopige versie \today
}} \hfill \includegraphics[width=2cm]{logo_ijkingstoets.jpg}}
\fancyfoot{}

\newwrite\idfile
\immediate\openout\idfile=ir5_IDreeks1.tex
\newwrite\classfile
\immediate\openout\idfile=ir5_CLASSreeks1.tex
\newwrite\catfile
\immediate\openout\idfile=ir5_CATreeks1.tex

\begin{document}

\Opensolutionfile{ans}[ans1]
\Opensolutionfile{numans}[2015_ir6_SLEUTELreeks1]

\input{feedback}
\newpage
\newcommand{\bron}[1]{\begin{scriptsize} \emph{#1} \end{scriptsize}}
%\newcommand{\id}[1]{\begin{scriptsize} id: \emph{#1} \end{scriptsize}}  
\newcommand{\id}[1]{\mbox{ } \immediate\write\idfile{#1}} 
\newcommand{\cat}[1]{categorie: #1\\}  
\newcommand{\class}[1]{ingeschatte moeilijkheidsgraad: #1\\} 
\newcommand{\info}[1]{#1\\}  
\newcommand{\juist}[1]{juist beantwoord: #1 \% \\}  
\newcommand{\blanco}[1]{blanco: #1 \% \\}  
\newcommand{\ul}[1]{#1}
\newcommand{\opm}[1]{opmerking vooraf: #1}

\input{2016beg09_stat}
\input{2016mod02_stat}
\newpage
\input{2016beg08_stat}
\input{2016red04_stat}
\newpage


\input{2016wis09_stat}
\input{2016red09_stat}
\input{2016wis04_stat}
\input{2016red02_stat}

\newpage


\input{2016beg05_stat}
\input{2016wis05_stat}
\newpage
\input{2016beg13_stat}
\input{2016mod04_stat}

\newpage

\input{2016wis08_stat}
\input{2016beg01_stat}

\newpage
\input{2016mod03_stat}
\input{2016red07_stat}
\newpage
\input{2016wis12_stat}
\input{2016mod05_stat}

\newpage


\input{2016red08_stat}
\input{2016wis13_stat}
\newpage
\input{2016mod07_stat}
\input{2016wis01_stat}
\newpage
\input{2016mod08_stat}
\input{2016beg02_stat}

\newpage 

\input{2016beg04_stat}
\input{2016red01_stat}
\newpage
\input{2016red06_stat}
\input{2016beg07_stat}
\newpage
\input{2016mod01_stat}
\input{2016wis02_stat}
\newpage
\input{2016ri01_stat}
\input{2016ri02_stat}



\end{document}


